%%%%%%%%%%%%%%%%%%%%%%% file template.tex %%%%%%%%%%%%%%%%%%%%%%%%%
%
% This is a general template file for the LaTeX package SVJour3
% for Springer journals.          Springer Heidelberg 2010/09/16
%
% Copy it to a new file with a new name and use it as the basis
% for your article. Delete % signs as needed.
%
% This template includes a few options for different layouts and
% content for various journals. Please consult a previous issue of
% your journal as needed.
%
%%%%%%%%%%%%%%%%%%%%%%%%%%%%%%%%%%%%%%%%%%%%%%%%%%%%%%%%%%%%%%%%%%%
%
% First comes an example EPS file -- just ignore it and
% proceed on the \documentclass line
% your LaTeX will extract the file if required
\begin{filecontents*}{example.eps}
%!PS-Adobe-3.0 EPSF-3.0
%%BoundingBox: 19 19 221 221
%%CreationDate: Mon Sep 29 1997
%%Creator: programmed by hand (JK)
%%EndComments
gsave
newpath
  20 20 moveto
  20 220 lineto
  220 220 lineto
  220 20 lineto
closepath
2 setlinewidth
gsave
  .4 setgray fill
grestore
stroke
grestore
\end{filecontents*}
%
\RequirePackage{fix-cm}
%
%\documentclass{svjour3}                     % onecolumn (standard format)
%\documentclass[smallcondensed]{svjour3}     % onecolumn (ditto)
\documentclass[smallextended]{svjour3}       % onecolumn (second format)
%\documentclass[twocolumn]{svjour3}          % twocolumn
%
\smartqed  % flush right qed marks, e.g. at end of proof
%
\usepackage{graphicx}
%
% \usepackage{mathptmx}      % use Times fonts if available on your TeX system
%
% insert here the call for the packages your document requires
%\usepackage{latexsym}
% etc.
%
% please place your own definitions here and don't use \def but
% \newcommand{}{}
%
% Insert the name of "your journal" with
% \journalname{myjournal}
%
\begin{document}

\title{Insert your title here%\thanks{Grants or other notes
%about the article that should go on the front page should be
%placed here. General acknowledgments should be placed at the end of the article.}
}
\subtitle{Do you have a subtitle?\\ If so, write it here}

%\titlerunning{Short form of title}        % if too long for running head

\author {\textbf{Bobak Farzin$^1$}, \textbf{Nombre Apellidos2$^2$}\\
	$^1$Universidad o lugar de trabajo\\
	$^2$Universidad o lugar de trabajo\\
	bfarzin@gmail.com\\
}

%\author{Bobak Farzin         \and
%        Second Author %etc.
%}

%\authorrunning{Short form of author list} % if too long for running head

%\institute{F. Author \at
%              first address \\
%              Tel.: +123-45-678910\\
%              Fax: +123-45-678910\\
%              \email{bfarzin@gmail.com}           %  \\
%             \emph{Present address:} of F. Author  %  if needed
%           \and
%           S. Author \at
%              second address
%}

\date{Received: 15 June 2019 / Accepted: date}
% The correct dates will be entered by the editor


\maketitle

\begin{abstract}
Insert your abstract here. Include keywords, PACS and mathematical
subject classification numbers as needed.
\keywords{First keyword \and Second keyword \and More}
% \PACS{PACS code1 \and PACS code2 \and more}
% \subclass{MSC code1 \and MSC code2 \and more}
\end{abstract}

\section{Introduction}
\label{intro}
Included citations as necessary: ~\cite{Nobody06}
\paragraph{Contribution} Our contribution with this work is..

\section{Task and Dataset Description}
\label{sec:task}
The \textit{Humor Analysis based on Humor Annotation (HAHA)} competition asked for analysis of two tasks in the Spanish language based on a coprpus of publicly collected data ~\cite{castro2018crowd}:
\begin{itemize}
\item \textbf{Task1: Humor Detection}:Detemine if a tweet is humorous. System ranking is based on F1 score which will balance precision and accuracy.
\item \textbf{Task2: Funniness Score}:If humorous, what is the average humor rating of the tweet. System ranking is based on RMSE.
\end{itemize}
HAHA dataset includes labeled data for 24,000 tweets and a test set of 6,000 tweets (80\%/20\% train/test split.)  Each record includes the raw tweet text (including accents and emoticons) and a True/False labels as well as a ``Funninness Score'' that is the average of the 1 to 5 start votes cast.  Examples and data can be found on the CodaLab competition webpage\footnote{http://competitions.codalab.org/competitions/22194/}.

\section{System Description}
\label{sec:system}

We generally follow the method of ULMFiT ~\cite{DBLP:journals/corr/abs-1801-06146}
\begin{enumerate}
	\item Train a language model (LM) on a large corpus of data
	\item Fine-tune the LM based on the target task language data
	\item Replace the final layer of the LM with a softmax or linear output layer and then fine-tune on the particular task at hand (classification or regression)
\end{enumerate}
Below we will give more detail on each step and the parameters used to generate our system.
\subsection{Data, Cleanup \& Tokenization}
\label{sec:datacleaning}
\subsection{Additional Data}
For our initial training, we  collected 475,143 tweets in the Spanish language using tweepy ~\cite{tweepy}.  The vocabulary and frequency of terms, punctuation and vocabulary can be quite different from the standard Wikipedia corpus that is used to train.

We combined the labeled and un-labeled text data so that we have the largest corpus of language for our fine-tuning step. 

\subsection{Cleaning}
We applied a list of default cleanup functions included in Fastai and added an additional one for this Twitter dataset.
\begin{itemize}
	\item Add spaces between special chars (ie. \verb|!!!| to \verb|! ! !|)
	\item Remove useless spaces (remove more than 2 spaces in sequence)
	\item Replace repetition at the character level (ie. \verb|grrrreat| becomes \verb|g xxrep r 3 eat|)
	\item Replace repition at the word level (similar to above)
	\item Deal with ALL CAPS words replacing with a token and converting to lower case.
	\item \textbf{NEW:} Move all text onto a single line by replacing new-lines inside a tweet with a reserved word (ie. \verb|\n| to \verb|xxnl|)
\end{itemize} 

\pagebreak  %move as needed to keep tweet text together
Here is an example of replacing the original tweet with our parsed version.

Original:
\begin{verbatim} 
Saber, entender y estar convencides que la frase \
#LaESILaDefendemosEntreTodes es nuestra linea es nuestro eje.\
#AlertaESI!!!!
Vamos por mas!!! e invitamos a todas aquellas personas que quieran \
se parte.
\end{verbatim}

Cleaned up:
\begin{verbatim} 
xxbos saber , entender y estar convencides que la frase \
# laesiladefendemosentretodes es nuestra linea es nuestro eje.\
xxnl  # alertaesi xxrep 4 ! xxnl vamos por mas ! ! ! e invitamos a \
todas aquellas personas que quieran se parte.
\end{verbatim}

\subsection{Tokenization}
We used sentencepiece~\cite{DBLP:journals/corr/abs-1808-06226} to parse into sub-word units and reduce the possible out-of-vocabulary (OOV) terms in the data set.  We selected a vocab size of 30,000 sub-word units and got 99.95\% character coverage including emojis.

\section{Training and Results}
\label{sec:4}
LM: 10\% validation set for training


oversample minority class to balance for better training using SMOTE \cite{Chawla:2002:SSM:1622407.1622416}

\subsection{Random Seed as a Hyperparamter}

\section{Conclusion}
\label{sec:5}

\section*{Acknowlegements}
% For one-column wide figures use
\begin{figure}
% Use the relevant command to insert your figure file.
% For example, with the graphicx package use
  \includegraphics{example.eps}
% figure caption is below the figure
\caption{Please write your figure caption here}
\label{fig:1}       % Give a unique label
\end{figure}
%
% For two-column wide figures use
\begin{figure*}
% Use the relevant command to insert your figure file.
% For example, with the graphicx package use
  \includegraphics[width=0.75\textwidth]{example.eps}
% figure caption is below the figure
\caption{Please write your figure caption here}
\label{fig:2}       % Give a unique label
\end{figure*}
%
% For tables use
\begin{table}
% table caption is above the table
\caption{Please write your table caption here}
\label{tab:1}       % Give a unique label
% For LaTeX tables use
\begin{tabular}{llll}
\hline\noalign{\smallskip}
first & second & third & fourth \\
\noalign{\smallskip}\hline\noalign{\smallskip}
number & number & number \\
number & number & number \\
\noalign{\smallskip}\hline
\end{tabular}
\end{table}


%\begin{acknowledgements}
%If you'd like to thank anyone, place your comments here
%and remove the percent signs.
%\end{acknowledgements}


% Authors must disclose all relationships or interests that 
% could have direct or potential influence or impart bias on 
% the work: 
%
% \section*{Conflict of interest}
%
% The authors declare that they have no conflict of interest.


% BibTeX users please use one of
\bibliographystyle{spbasic}      % basic style, author-year citations
%\bibliographystyle{spmpsci}      % mathematics and physical sciences
%\bibliographystyle{spphys}       % APS-like style for physics
\bibliography{local}   % name your BibTeX data base

\end{document}
% end of file template.tex

