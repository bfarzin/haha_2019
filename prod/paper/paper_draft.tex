%%%%%%%%%%%%%%%%%%%%%%% file template.tex %%%%%%%%%%%%%%%%%%%%%%%%%
%
% This is a general template file for the LaTeX package SVJour3
% for Springer journals.          Springer Heidelberg 2010/09/16
%
% Copy it to a new file with a new name and use it as the basis
% for your article. Delete % signs as needed.
%
% This template includes a few options for different layouts and
% content for various journals. Please consult a previous issue of
% your journal as needed.
%
%%%%%%%%%%%%%%%%%%%%%%%%%%%%%%%%%%%%%%%%%%%%%%%%%%%%%%%%%%%%%%%%%%%
%

%
\RequirePackage{fix-cm}
%
%\documentclass{svjour3}                     % onecolumn (standard format)
%\documentclass[smallcondensed]{svjour3}     % onecolumn (ditto)
\documentclass[smallextended]{svjour3}       % onecolumn (second format)
%\documentclass[twocolumn]{svjour3}          % twocolumn
%
\smartqed  % flush right qed marks, e.g. at end of proof
%
\usepackage{graphicx}
%
% \usepackage{mathptmx}      % use Times fonts if available on your TeX system
%
% insert here the call for the packages your document requires
%\usepackage{latexsym}
% etc.
%
% please place your own definitions here and don't use \def but
% \newcommand{}{}
%
% Insert the name of "your journal" with
% \journalname{myjournal}
%
\begin{document}

\title{Insert your title here%\thanks{Grants or other notes
%about the article that should go on the front page should be
%placed here. General acknowledgments should be placed at the end of the article.}
}
%\subtitle{Do you have a subtitle?\\ If so, write it here}

%\titlerunning{Short form of title}        % if too long for running head

\author {\textbf{Bobak Farzin$^1$}, \textbf{Nombre Apellidos2$^2$}\\
	$^1$bfarzin@gmail.com\\
	$^2$Universidad o lugar de trabajo\\
}

%\author{Bobak Farzin         \and
%        Second Author %etc.
%}

%\authorrunning{Short form of author list} % if too long for running head
\institute{}
%\institute{F. Author \at
%              first address \\
%              Tel.: +123-45-678910\\
%              Fax: +123-45-678910\\
%              \email{bfarzin@gmail.com}           %  \\
%             \emph{Present address:} of F. Author  %  if needed
%           \and
%           S. Author \at
%              second address
%}

\date{Received: 15 June 2019 / Accepted: date}
% The correct dates will be entered by the editor


\maketitle

\begin{abstract}
Our entry into the \textit{HAHA 2019 Challenge} placed $3^{rd}$ in the classification task and $2^{nd}$ in the regression task.  We describe our system and innovations, as well as comparing our results to a Naive Bayes baseline. All the code for our project is included in a GitHub\footnote{https://github.com/bfarzin/haha\_2019\_final} repository for easy reference and to enable replication by others.

\keywords{First keyword \and Second keyword \and More}
% \PACS{PACS code1 \and PACS code2 \and more}
% \subclass{MSC code1 \and MSC code2 \and more}
\end{abstract}

\section{Introduction}
\label{intro}
\newcommand{\chapquote}[3]{\begin{quotation} \textit{#1} \end{quotation} \begin{flushright} - #2, \textit{#3}\end{flushright} }

\begin{quote}
- !`Socorro, me ha picado una víbora!\\
- ?`Cobra?\\
- No, gratis.\footnote{https://www.fluentin3months.com/spanish-jokes/}
\end{quote}
Google Translation:
\begin{quote}
- Help, I was bitten by a snake!\\
- Does it charge?\\
- Not free.
\end{quote}
Humor does not translate well because it often relies on double-meaning or a subtle play on word choice, pronunciation, or context.  These issues are further exacerbated in areas where space is a premium (as frequent on social media platforms), often leading to usage and development of shorthand, in-jokes, and self-reference. Thus, building a system to classify the humor of tweets is a difficult task.  However, with transfer-learning and the Fastai library\footnote{https://docs.fast.ai/}, we show the power of these tools in order to build a high quality classifier in a foreign language. Our system strongly outperforms a Naive Bayes SVM baseline, which is frequently considered a "strong baseline" for many NLP related tasks \footnote{https://twitter.com/radamihalcea/status/1119020144389443585}.

Rather than hand-crafted language features, we have taken an "end to end" approach building from the raw text to a final model that achieve the tasks as presented.  Our paper lays out the details of the system and our code can be found in a GitHub repo for use by other researchers to extend the state of the art in sentiment analysis. 

\paragraph{Contribution} Our key contributions are three fold.  First, we apply transfer-learning of a language model based on a larger corpus of tweets.  Second, we use a label smoothed loss to allow full training of the final model without gradual unfreezing.  Third, we select the best model for each task based on cross-validation and 20 random-seed initialization in the final network training step.

\section{Task and Dataset Description}
\label{sec:task}
The \textit{Humor Analysis based on Humor Annotation (HAHA) 2019}\cite{overview_haha2019} competition asked for analysis of two tasks in the Spanish language based on a corpus of publicly collected data ~\cite{castro2018crowd}:
\begin{itemize}
\item \textbf{Task1: Humor Detection}:Determine if a tweet is humorous. System ranking is based on F1 score which balances precision and accuracy.
\item \textbf{Task2: Funniness Score}:If humorous, what is the average humor rating of the tweet? System ranking is based on RMSE.
\end{itemize}
The HAHA dataset includes labeled data for 24,000 tweets and a test set of 6,000 tweets (80\%/20\% train/test split.)  Each record includes the raw tweet text (including accents and emoticons), a binary humor label, the number of votes for each of five star ratings and a ``Funniness Score'' that is the average of the 1 to 5 star votes cast.  Examples and data can be found on the CodaLab competition webpage\footnote{http://competitions.codalab.org/competitions/22194/}.

\section{System Description}
\label{sec:system}
We generally follow the method of ULMFiT ~\cite{HowardRuder:DBLP:journals/corr/abs-1801-06146} including pre-training and differential learning rates. 
\begin{enumerate}
	\item Train a language model (LM) on a large corpus of data
	\item Fine-tune the LM based on the target task language data
	\item Replace the final layer of the LM with a softmax or linear output layer and then fine-tune on the particular task at hand (classification or regression)
\end{enumerate}
Below we will give more detail on each step and the parameters used to generate our system.
\subsection{Data, Cleanup \& Tokenization}
\label{sec:datacleaning}
\subsection{Additional Data}
For our initial training, we  collected 475,143 tweets in the Spanish language using tweepy ~\cite{Tweepy}.  The frequency of terms, punctuation and vocabulary can be quite different from the standard Wikipedia corpus that is often used to train an LM.

In the fine-tuning step, we combined the labeled and un-labeled text data for the LM training.
\subsection{Cleaning}
We applied a list of default cleanup functions included in Fastai\cite{} and added an additional one for this Twitter dataset.
\begin{itemize}
	\item Add spaces between special chars (ie. \verb|!!!| to \verb|! ! !|)
	\item Remove useless spaces (remove more than 2 spaces in sequence)
	\item Replace repetition at the character level (ie. \verb|grrrreat| becomes \verb|g xxrep r 3 eat|)
	\item Replace repetition at the word level (similar to above)
	\item Deal with ALL CAPS words replacing with a token and converting to lower case.
	\item \textbf{Addition:} Move all text onto a single line by replacing new-lines inside a tweet with a reserved word (ie. \verb|\n| to \verb|xxnl|)
\end{itemize} 
%\pagebreak  %move as needed to keep tweet text together
The following example shows the application of this data cleaning to a single tweet:
\begin{verbatim} 
Saber, entender y estar convencides que la frase \
#LaESILaDefendemosEntreTodes es nuestra linea es nuestro eje.\
#AlertaESI!!!!
Vamos por mas!!! e invitamos a todas aquellas personas que quieran \
se parte.
\end{verbatim}

\begin{verbatim} 
xxbos saber , entender y estar convencides que la frase \
# laesiladefendemosentretodes es nuestra linea es nuestro eje.\
xxnl  # alertaesi xxrep 4 ! xxnl vamos por mas ! ! ! e invitamos a \
todas aquellas personas que quieran se parte.
\end{verbatim}

\subsection{Tokenization}
We used sentencepiece~\cite{SentencePiece:DBLP:journals/corr/abs-1808-06226} to parse into sub-word units and reduce the possible out-of-vocabulary (OOV) terms in the data set.  We selected a vocab size of 30,000 and used the byte-pair encoding (bpe) model. 

\section{Training and Results}
\label{sec:4}
\subsection{LM Training and Fine-tuning}
We train the LM using a 90/10 training/validation split, reporting the validation loss and accuracy of next-word prediction on the validation set. For the LM, we selected a AWD\_LSTM \cite{Merity:DBLP:journals/corr/abs-1708-02182} model included in Fastai with QRNN\cite{Bradbury:DBLP:journals/corr/BradburyMXS16} units, 2304 hidden-states, 3 layers and a softmax layer to predict the next-word.  We tied the embedding weights on the encoder and decoder for training.  We performed some simple tests with LSTM units and a Transformer Language model, finding all models were similar in performance during LM training. We thus chose to use QRNN units due to improved training speed compared to the alternatives. This model has about 60 million trainable parameters.  

Our loss is label smoothing\cite{Labelsmoothing:DBLP:journals/corr/PereyraTCKH17} of the flattened cross-entropy loss. 
Parameters used for training and finetuning are shown in Table \ref{tab:tab_training}.
For all networks we applied a dropout multiplier which scales the dropout used throughout the network.  We used the Adam optimizer with weight decay as indicated in the table.  

Following the work of Smith\cite{Smith:DBLP:journals/corr/abs-1803-09820}  we found the largest learning-rate that we could apply and then ran a one-cycle policy for a single epoch. We then ran subsequent training with one-cycle and lower learning rates indicated in Table \ref{tab:tab_training}.

\begin{table}[ht]
	% table caption is above the table
	\caption{LM Training Parameters}
	\label{tab:tab_training}       % Give a unique label
\begin{tabular}{lll}
	\hline\noalign{\smallskip}
	Param & LM & Fine-Tune LM \\
	\noalign{\smallskip}\hline\noalign{\smallskip}
	Weight Decay & 0.1 & 0.1 \\
	Dropout Mult & 1.0 & 1.0 \\
	Learning Rate & 1 epoch at $5*10^{-3}$ & 5 epochs at $3*10^{-3}$ \\
    Cont. Training & 15 epochs at $1*10^{-3}$ & 10 epochs at $1*10^{-4}$\\
	\noalign{\smallskip}\hline
\end{tabular}
\end{table}

\subsection{Classification and Regression Fitting}
Again, following the play-book from \cite{HowardRuder:DBLP:journals/corr/abs-1801-06146}, we change pre-trained network head to a softmax or linear output layer (as appropriate for the transfer task) and then load the LM weights for the layers below.  We train just the new head from random initialization, then unfreeze the entire network and train with differential learning rates.

With the same learning rate and weight decay we apply a 5-fold cross-validation on the outputs and take the mean across the folds as our ensemble.  We sample 20 random seeds (see more in section \ref{sec:rand_seeds}) to find the best initialization for our gradient descent search.  From these samples, we select the best validation F1 metric or MSE for use in our test submission.
\subsubsection{Classifier setup}  For the classifier, we have a softmax head and label smoothing loss which allows us to train without gradual unfreezing.  We oversample the minority class to balance the outcomes for better training using SMOTE \cite{Chawla:2002:SSM:1622407.1622416}.  
\subsubsection{Regression setup}  For the regression task, we fill all \verb|#N/A| labels with scores of 0.  We add a linear head output and mean-squared-error (MSE) loss function. 

\begin{table}[ht]
	% table caption is above the table
	\caption{Classification and Regression Training Parameters}
	\label{tab:clas_training}       % Give a unique label
	\begin{tabular}{ll}
		\hline\noalign{\smallskip}
		Param & Value \\
		\noalign{\smallskip}\hline\noalign{\smallskip}
		Weight Decay & 0.1  \\
		Dropout Mult &  0.7 \\
		Learning Rate (Head)& 2 epochs at $1*10^{-2}$\\
		Cont. Training & 15 epochs with diff lr:($1*10^{-3}/(2.6^4)$, $5*10^{-3}$)\\
		\noalign{\smallskip}\hline
	\end{tabular}
\end{table}

\subsection{Random Seed as a Hyperparamter}
\label{sec:rand_seeds}
For classification and regression, the random seed sets the initial random weights of the head layer. This initialization affects the final F1 metric achievable.  

Across each of the 20 random seeds, we average the 5-folds and obtain a single F1 metric on the validation set. The histogram of 20-seed outcomes is shown in Figure \ref{fig:random_seed_hist} and covers a range  from 0.820 to 0.825 over the validation set. We selected our single best random seed for the test submission. With more exploration, a better seed could likely be found.  Though we only use a single seed for the LM training, one could do a similar search with random seeds for LM pretraining, and further select the best down-stream seed similar to \cite{poleval}

\begin{figure}[ht]
	% Use the relevant command to insert your figure file.
	% For example, with the graphicx package use
	\includegraphics[width=0.75\textwidth]{seed_hist_f1}
	\caption{Histogram of F1 metric averaged across 5-fold metric}
	\label{fig:random_seed_hist}
\end{figure}

\subsection{Results}
Below in Table \ref{tab:tab_results} we layout three results from our submissions in the competition. The first is the baseline Naive Bayes SVM solution, with an F1 of 0.7548.  Second is our first random seed selected for the classifier which produces a 0.8083 result.  While better than the NB-SVM solution, we can do better by picking the best F1 from the 20 seeds that we selected. That result is in the final row and resulting in our final submission value of 0.8099. Note that applying this model over our validation set achieved an F1 of 0.8254 however our final test set is slightly lower, a drop of 0.0155 to the final test F1 of 0.8099.

\begin{table}[ht]
	% table caption is above the table
	\caption{Comparative Results}
	\label{tab:tab_results}
	\begin{tabular}{lllll}
		\hline\noalign{\smallskip}
		 & Accuracy & Precision & Recall & F1 \\
		\noalign{\smallskip}\hline\noalign{\smallskip}
NBSVM &      0.8223 & 0.8180 & 0.7007 & 0.7548 \\
First Seed & 0.8461 & 0.7869 & 0.8309 & 0.8083 \\
Best Seed &  0.8458 & 0.7806 & 0.8416 & \textbf{0.8099} \\
		\noalign{\smallskip}\hline
	\end{tabular}
\end{table}

\section{Conclusion}
\label{sec:5}
This paper describes our implementation of a neural net model for classification and regression in the HAHA 2019 challenge.  Our solution placed third in Task 1 and second in task 2 in the final competition standings.  We describe the data collection, pre-training, and final model building steps for this contest.  Additionally, we have open-sourced all  code used in this contest to further enable research on this task in the future.

\begin{acknowledgements}
If you'd like to thank anyone, place your comments here
and remove the percent signs.
\end{acknowledgements}


% Authors must disclose all relationships or interests that 
% could have direct or potential influence or impart bias on 
% the work: 
%
% \section*{Conflict of interest}
%
% The authors declare that they have no conflict of interest.


% BibTeX users please use one of
%\bibliographystyle{fullname}      % basic style, author-year citations

%\bibliographystyle{spbasic}      % basic style, author-year citations
\bibliographystyle{spmpsci}      % mathematics and physical sciences
%\bibliographystyle{spphys}       % APS-like style for physics
\bibliography{local}   % name your BibTeX data base

\end{document}
% end of file template.tex

